\section{Message Definition -- PDU}
\label{sec:pdus}

In this section, we discuss the details of the \textsf{RCUC} protocol.

\subsection{Addressing}
\label{sec:pdus:addr}
\textsf{RCUC} protocol uses TCP/IP for transportation. It uses stream-oriented channels to send messages with no fixed sizes. [Note: We can customize port numbers to for specific task done by a device on the car. For example: We may want to assign base\_port\_number + device\_port\_index ex: 1010+1 = 1011 port no for air conditioner. ].

Based on port configuration we explained, a TCP connection is established to the \textsf{CCS} IP address and port number provided.
\subsection{Flow Control}
\label{sec:pdus:flow}
Flow control is dispatched to TCP/IP layer. [Note: As future extension, we may mention that we can enhance the flow control with adding asychrony in communication such as when client sends a message  and returns but sender send feedback later on. This reduces the congestion of network. In addition, we may buffer requests and flush them to server to save reduce communication overhead for each command.]
 
\subsection{PDU Definitions}
\label{sec:pdus:pdu}
\begin{itemize}
\item \textsf{Handshake}: is done via authentication.
\item \textsf{Utility Commands}: These are commands in between client and server. Client sends a command and waits for a response from server.
  \end{itemize}

Our PDU definitions are based on generic chunk design:
\[\textsf{Identifier/Code} : \textsf{Size}\]

Each command is going to be in the following structure:
\[\textsf{[Command Code:Size]} \;\; \textsf{[Number Of Chunks:Size]}\]
where \textsf{[Command Code:Size]} denotes the command type with a fixed size, 1 byte and \textsf{[Number Of Chunks:Size]} denotes the number of following chunks used to form the command. 
\subsubsection{Handshake}
\label{sec:pdus:pdu:hs}
All the handshaking is handled with underlying transport layer protocol. What we additionally provide as a part of handshaking is authentication of an user and we expalin it in the following sections.
\subsubsection{Client to Server Messages}
\label{sec:pdus:pdu:c_to_s}

\paragraph{User Authentication}: Each user is identifier with its user-id. Each user needs to authenticate to choose a car and request utility functions to be applied to the car's devices.
\begin{table}[ht!]
  \centering

\begin{tabular}{l l l l}
\hline
\textbf{Command Code} & \textbf{Number Of Chunks} &\textbf{User ID} &   \textbf{User Password}  \\
\hline

\hline
\textsf{[0x03:1]} -- Authentication Command  & \textsf{[0x02:1]} -- Number of Chunks & \textsf{[0x00:1]} -- User 1 & \textsf{[0x00:1]} -- Password 1   \\
\hline
                 & \textsf{[0x01:1]} -- User 2 & \textsf{[0x12:1]} -- Password 2 \\
\hline
                 & \textsf{[0x02:1]}  -- User 3 & \textsf{[0x30:1]} -- Password 3\\
\hline
\hline
\end{tabular}
\caption{List of users and their encodings.}
\end{table}

An example usage of this command is :
\[\textsf{client} --> \textsf{[0X03:1]} \;\;\textsf{[0x02:1]} \;\;\textsf{[0x00:1]} \;\; \textsf{[0x00:1]}\]
where \textsf{[0x03]} denotes the type of command, \textsf{Authentication Command}, \textsf{0x02} denotes number of chunk following which is 2, the first \textsf{[0x00:1]} denotes user id and the second \textsf{[0x00:1]} denotes the password of the user.

\paragraph{Car Utility Request}: Once an user is authenticated and accessed its car, he can start change the car's devices utility functions via client which sending request to server for these actions.
\begin{table}[ht!]
  \centering
  \resizebox{\textwidth}{!}{
    
\begin{tabular}{l l l l l l}
\hline
\textbf{Command Code} &\textbf{Number of Chunks} & \textbf{Vehicle ID} &\textbf{Device ID}  & \textbf{States Code} &  \textbf{Actions Code} \\
\hline
\hline

\textsf{[0x05:1]} -- Utility Request Command & \textsf{0x03} -- Number of Chunks  & \textsf{0x01} -- Car ID &  \textsf{[0x00:1]} -- Radio    &  \textsf{[0x00:1]} -- on   & \textsf{[0x00:1]} -- turn on  \\
                                   & \textsf{[0x01:1]} --  off  & \textsf{[0x01:1]} -- turn off\\
                                   &                            & \textsf{[0x02:1]} -- volume up\\
                                   &                            & \textsf{[0x03:1]} -- volume on\\
\hline
     & \textsf{[0x01:1]} -- Doors    & \textsf{[0x00:1]} -- locked     & \textsf{[0x00:1]} -- lock  \\
                                    & \textsf{[0x01:1]} -- unlocked   & \textsf{[0x01:1]} -- unlock\\
\hline
  &\textsf{[0x02:1]} -- Window    & \textsf{[0x00:1]} -- left   & \textsf{[0x00:1]} -- up  \\
                                 & \textsf{[0x01:1]} -- right  & \textsf{[0x01:1]} -- down\\
                                 &                             & \textsf{[0x02:1]} -- left\\
                                 &                             & \textsf{[0x03:1]} -- right\\

\hline
   & \textsf{[0x03:1]} -- AC  & \textsf{[0x00:1]} -- on  & \textsf{[0x00:1]} -- turn on \\
                            & \textsf{[0x01:1]} -- off & \textsf{[0x01:1]} -- turn off\\
                            &                          & \textsf{[0x02:1]} -- increase\\
                            &                          & \textsf{[0x03:1]} -- decrease\\
\hline
\hline
\end{tabular}}
\caption{List of devices and their encodings for a vehicle.}
\end{table}


\textsf{Command Code}, $\textsf{[0x05:1]}$, denotes an identifier/code for a command which is used for requesting a utility to be applied on device. \textsf{Device ID}, $\textsf{[0x00:1]}$ denotes an unique identifier/code used for a device, ex: \textsf{Radio}. \textsf{States Code}, $\textsf{[0x00:1]}$ denotes an identifier/code for keeping the status of the device, \textsf{off-on}. \textsf{Actions Code}, $\textsf{0x00:1}$, denotes an identifier/code for showing the action to be applied on device, ex: \textsf{turn-off, turn-on, vol\_up, vol\_down}.  


An example usage of utility command request from client is
\[\textsf{client} --> \textsf{[0x05:1]} \;\; \textsf{[0x03]} \;\; \textsf{[0X01]} \;\;  \textsf{[0x00:1]}   \;\;  \textsf{[0x00:1]}\]
where \textsf{[0x05:1]} is utility request command encoding , \textsf{[0x03:1]} is the number of chunks following in the command, \textsf{[0x01:1]} is vehicle identifier, the following  \textsf{[0x00:1]} is device identifier and the last one,  \textsf{[0x00:1]}, is the encoding of the action to be performed on device which is turning on the radio.

\paragraph{Termination}: Termination is the last action which can be performed by both client and server. Once command for termination is sent by any of two, all resources being used for the connection is properly finalized and connection is terminated.

\begin{table}[ht!]
  \centering
  \begin{tabular}{l l}
\hline
\textbf{Command Code} & \textbf{[Number Of Chunks]}\\
\hline
\hline
\textsf{[0x09:1]} -- Terminate Command & \textsf{0x00} -- Number of Chunks  \\
\hline
\hline
\end{tabular}
\caption{Termination command sent by Client/Server}
\end{table}

\subsubsection{Server to Client Messages}
\label{sec:pdus:pdu:s_to_c}
There are two cases where server updates client:
\begin{itemize}
\item When client sent an command and the server updates with a confirmation that the action is applied or an error message.
  \end{itemize}

When a message is received from a server then its well-formedness verification is performed.
\begin{itemize}
\item  We can enforce a type/structure for device type identifier.
\item We can enforce a type/structure for device number.
\item  We can enforce constraints on which state received-command is legal to execute.
\item Received parameters/values are legal for the command to be executed. 
\end{itemize}

Server sends an update message :

\begin{table}[ht!]
  \centering
  \begin{tabular}{l l l}
\hline
\textbf{Command Code} & \textbf{Number Of Chunks} & \textbf{Content}\\
\hline
\hline
\textsf{[0x06:1]} -- Failure Command  & \textsf{[0x01]} -- Number of Chunks & \textsf{[Failure\_String:1]} \\
\hline
\hline
\end{tabular}
\caption{Server Failure Message}
\end{table}


For the positive feedbacks to clients requests:

\begin{table}[ht!]
  \centering
\begin{tabular}{l l l}
\hline
\textbf{Command Code}  & \textbf{Number Of Chunks} & \textbf{Content}\\
\hline
\hline
\textsf{[0x07:1]} -- Success Command & \textsf{[0x01]} -- Number of Chunks & \textsf{[Success\_String:1]} \\
\hline
\hline
\end{tabular}
\caption{Server Success Message}
\end{table}


\subsection{Error Control}
\label{sec:pdus:err}
Error messages are raised when
\begin{itemize}
\item Server checks request and raise error if request message is not verified.
\item Authentication is not provided
  \end{itemize}
