\section{Message Definition -- PDU}
\label{sec:pdus}

In this section, we discuss the details of the \textsf{RCUC} protocol.

\subsection{Addressing}
\label{sec:pdus:addr}
\textsf{RCUC} protocol uses TCP/IP for transportation. It uses stream-oriented channels to send messages with no fixed sizes. [Note: We can customize port numbers to for specific task done by a device on the car. For example: We may want to assign base\_port\_number + device\_port\_index ex: 1010+1 = 1011 port no for air conditioner. ].

Based on port configuration we explained, a TCP connection is established to the \textsf{CCS} IP address and port number provided.
\subsection{Flow Control}
\label{sec:pdus:flow}
Flow control is dispatched to TCP/IP layer. [Note: As future extension, we may mention that we can enhance the flow control with adding asychrony in communication such as when client sends a message  and returns but sender send feedback later on. This reduces the congestion of network. In addition, we may buffer requests and flush them to server to save reduce communication overhead for each command.]
 
\subsection{PDU Definitions}
\label{sec:pdus:pdu}
\begin{itemize}
\item \textsf{Handshake}: is done via authentication.
\item \textsf{Utility Commands}: These are commands in between client and server. Client sends a command and waits for a response from server.
  \end{itemize}

\subsubsection{Handshake}
\label{sec:pdus:pdu:hs}

\subsubsection{Client to Server Messages}
\label{sec:pdus:pdu:c_to_s}


\begin{table}[ht!]
  \centering
  \resizebox{\textwidth}{!}{
\begin{tabular}{l l l l l l l l}
\hline
\textbf{Command Code} & \textbf{Command Name} & \textbf{Device ID} & \textbf{Device Name} & \textbf{State Name}& \textbf{States Code} & \textbf{Action Name}  & \textbf{Actions Code} \\
\hline
\hline

\textsf{[0x05]} & \textsf{ACTION\_CMD}    &     \textsf{0x00}  & \textsf{Radio}      & \textsf{off}  & \textsf{[0x00:1]}   & \textsf{turn\_on}   & \textsf{[0x00:1]}  \\
                &                         &                    &                     &  \textsf{on}  & \textsf{[0x01:1]}   & \textsf{turn\_off}  & \textsf{[0x01:1]} \\
                &                         &                    &                     &               &                     & \textsf{vol\_up}    & \textsf{[0x02:1]} \\
                &                         &                    &                     &               &                     &  \textsf{vol\_down} & \textsf{[0x03:1]} \\
\hline
\textsf{[0x05]} & \textsf{ACTION\_CMD}    &     \textsf{0x01}  & \textsf{Door}       &\textsf{unlocked}  & \textsf{[0x00:1]}   & \textsf{unlock}   & \textsf{[0x00:1]}  \\
                &                         &                    &                     &\textsf{locked}    & \textsf{[0x01:1]}   & \textsf{lock}  & \textsf{[0x01:1]} \\
\hline
\textsf{[0x05]} & \textsf{ACTION\_CMD}    &     \textsf{0x02}  & \textsf{Mirror}     &\textsf{left}     & \textsf{[0x00:1]}   & \textsf{move\_right} & \textsf{[0x00:1]}  \\
                &                         &                    &                     &\textsf{right}    & \textsf{[0x01:1]}   & \textsf{move\_left}  & \textsf{[0x01:1]} \\
                &                         &                    &                     &                  &                     & \textsf{move\_up}    & \textsf{[0x02:1]} \\
                &                         &                    &                     &                  &                     & \textsf{move\_down}  & \textsf{[0x03:1]} \\

\hline
\textsf{[0x05]} & \textsf{ACTION\_CMD}    &     \textsf{0x03}  & \textsf{AC}         &\textsf{off}      & \textsf{[0x00:1]}   & \textsf{turn\_on} & \textsf{[0x00:1]}  \\
                &                         &                    &                     &\textsf{on}       & \textsf{[0x01:1]}   & \textsf{turn\_off}  & \textsf{[0x01:1]} \\
                &                         &                    &                     &                  &                     & \textsf{increase}    & \textsf{[0x02:1]} \\
                &                         &                    &                     &                  &                     & \textsf{decrease}  & \textsf{[0x03:1]} \\
\hline
\hline
\end{tabular}}
\caption{Utility Commands}
\end{table}


\subsubsection{Server to Client Messages}
\label{sec:pdus:pdu:s_to_c}
There are two cases where server updates client:
\begin{itemize}
\item When client sent an command and the server updates with a confirmation that the action is applied or an error message.
  \end{itemize}

When a message is received from a server then its well-formedness verification is performed.
\begin{itemize}
\item \hl{[TODO: We can enforce a type/structure for device type identifier.]}
\item \hl{[TODO:We can enforce a type/structure for device number. ]}
\item \hl{[TODO: We can enforce constraints on which state received-command is legal to execute]}
  \item \hl{[TODO: Received parameters/values are legal for the command to be executed ]}
\end{itemize}

Server sends an update message :

\begin{table}[ht!]
  \centering
\begin{tabular}{l l l}
\hline
\textbf{Command Code} & \textbf{Command Name} & \textbf{Content}\\
\hline
\hline
\textsf{[0x06]} & \textsf{Server\_Error}  &     \textsf{[Command\_Applied\_Error\_String:1]} \\
\hline
\hline
\end{tabular}
\caption{Server Error Message}
\end{table}


For the positive feedbacks to clients requests:

\begin{table}[ht!]
  \centering
\begin{tabular}{l l l}
\hline
\textbf{Command Code} & \textbf{Command Name} & \textbf{Content}\\
\hline
\hline
\textsf{[0x07]} & \textsf{Server\_Update}  &     \textsf{[Command\_Applied\_String:1]} \\
\hline
\hline
\end{tabular}
\caption{Server Update Message}
\end{table}


\subsection{Error Control}
\label{sec:pdus:err}
Error messages are raised when
\begin{itemize}
\item Server checks request and raise error if request message is not verified.
\item Authentication is not provided
  \end{itemize}
