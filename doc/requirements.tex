%%%%%%%%%%%%%%%%
%%%  config  %%%
%%%%%%%%%%%%%%%%

\documentclass[12pt]{usenixsubmit}


\usepackage{ulem}


\usepackage{setspace}
\usepackage{verbatim}
\usepackage{graphicx}
\usepackage{amssymb}
\usepackage{array}
\usepackage{color}
\usepackage{hyperref}
\usepackage{listings}
\usepackage{adjustbox}
\newlength\someheight
\setlength\someheight{3cm}
\usepackage{fancyhdr}
\usepackage{enumitem}
\usepackage{geometry}
\usepackage{comment}

%%%%%%%%%%%%%%%%
%%%  config  %%%
%%%%%%%%%%%%%%%%

\pagestyle{fancy}

\geometry{letterpaper}

\hypersetup{
colorlinks,
linkcolor=blue,
filecolor=blue,
urlcolor=blue,
citecolor=blue}

\fancyhf{}
\fancyfoot[R]{\thepage}
\fancyfoot[L]{The Remote Automobile Utility Control Protocol $\mid$ Group 2}
\renewcommand{\headrulewidth}{0pt}
\renewcommand{\footrulewidth}{0pt}

\newcommand{\tuple}[1]{\ensuremath{\left \langle #1 \right \rangle }}
\newcommand{\sect}[1]{Sec.~\ref{#1}}
\newcommand{\figr}[1]{Fig.~\ref{#1}}
\newcommand{\tabl}[1]{Tab.~\ref{#1}}
\newcommand{\appn}[1]{App.~\ref{#1}}
\newcommand{\comm}[1]{}
\newcommand{\abs}[1]{\arrowvert#1\arrowvert}
\newcommand{\norm}[1]{\Arrowvert#1\Arrowvert}

\newcommand{\footlabel}[2]{%
    \addtocounter{footnote}{1}%
    \footnotetext[\thefootnote]{%
        \addtocounter{footnote}{-1}%
        \refstepcounter{footnote}\label{#1}%
        #2%
    }%
    $^{\ref{#1}}$%
}

\newcommand{\footref}[1]{%
    $^{\ref{#1}}$%
}

\usepackage{pgf}
\usepackage{tikz}

\usepackage{pgf-umlsd}
\usepackage{soul,color}

%\setstretch{0.9}
\usetikzlibrary{arrows,shadows,positioning,automata}
%%%%%%%%%%%%%%%%
%%% document %%%
%%%%%%%%%%%%%%%%

\begin{document}

%%% cover %%%

\title{Group 2: The Remote Automobile Utility Control Protocol Implementation Requirements \\ \Large{CS544 Spring 2017, Drexel University}}

\author{
    Lewis Cannalongo \\
    lrc74@drexel.edu
    \and
     Max Mattes \\
    mm3888@drexel.edu
    \and
    Ismail Kuru \\
    ik335@drexel.edu
}
\date{\today}

\maketitle
This document specifies the packages and files in the RAUC protocol implementation code base with explanation of the following requirements
\begin{itemize}
  \item {\tt STATEFUL}
  \item {\tt CONCURRENT}
  \item {\tt SERVICE}
  \item {\tt CLIENT}
  \item {\tt UI}
  \end{itemize}

The following sections include links to the explanations of the requirements in files at code base in git-hub repository.
\section{Client}
\subsection{Client.java}Purpose of this file is provide functionality to initialize a connection to the server. It uses GUI functinolity provided by ClientGUI.java to control the client side capabilities. 
\begin{itemize}
\item {\tt CLIENT} : requirement is satisfied by Client.java in a way that it provides instantiates Client object with parametrized Server port number and Server address. [ref to Client 50-58]
  \item {\tt CONCURRENT} : This requirement is satisfied by Client.java in a way that it spawns a Listener thread to process Servers responses [ref to Client 106-109] right after establishing connection with server [ref to Client 81-101] through SSL sockets.
\end{itemize}
           
\subsection{ClientGUI.java} Client.java handles interactions with user through ClientGUI.java object handle, [ref Client 35-36]. ClientGUI.java uses \textsf{javax.swing} GUI objects to handle user interaction and inputs [ref ClientGUI.java 23-34].
\begin{itemize}
\item {\tt UI} : ClientGUI.java abstracts the way communication is handled between Client and Server. A GUI based on \textsf{javax.swing} panels is provided to user for providing inputs which are used as arguments for buildiing a message to be sent to Server, [ref ClientGUI 102-112]. In addition, user authentication is provided with the inputs through Login-GUI [ref to ClientGUI.java 65-85]. 
 \item {\tt CLIENT} : is relavant to CLientGUI.java because this file enables you to indicate Server address and Server port number. [ref CLientGUI.java 40-60]
\end{itemize}

\section{Server}
\subsection{Server.java}This file is main driver for performing server functionalities.
\begin{itemize}
\item {\tt SERVICE} : Main class for starting a server. The server is hardcoded to port 1500. The server keeps client connections in a map [ref Server.java 23].
\item {\tt CONCURRENT} : The server initializes a connection listener which handles incoming connections, [ref 75-95]. It spawns a new thread, ConnectionThread , for handling each different client connection. This makes Server to handle multiple connections at a time , [ref Server.java 80-85].
\end{itemize}

\subsection{ServerGUI.java}This file includes implementation of GUI which aims to provide server's logs.
\begin{itemize}
\item {\tt UI} : Entire file's relevant requirement is satisfying UI.  
\end{itemize}

\subsection{ConnectionThread.java} This files includes the thread implementation aiming to handle server communication to the client. Main server drivers Server.java spawns ConnectionThread for each client communication.
\begin{itemize}
 \item {\tt SERVICE} : Main thread to handle server communication to the client, parse and respond to the client messages etc. [ref Server.java 180-198]
\end{itemize}
  
\section{DFAs}
\subsection{ClientDFASpec.java}This file includes the definitions for specification of the Client side functionality. It is sub type of DFASpec.java.
\begin{itemize}
\item  {\tt CLIENT} : Client side functionality is specified in this file.
\item {\tt STATEFUL} : Entire file defines valid client side states and admissible valid transitions between these states. All functions([ref ClientSpecDFA \textsf{sendInit}, receiveAuth, sendEstablished, receiveAuth, receiveWaitcmd, receiveWaitqry]) inherited from DFASpec.java defines the valid transition relation between client side states, setting state of client dfa in each of these functions, triggered by \textsf{Message m}. 
  \end{itemize}
\subsection{ServerDFASpec.java}This file includes the definitions for specification of the Server side functionality. It is sub type of DFASpec.java.
\begin{itemize}
  \item {\tt SERVICE} : Server side functionality is specified in this file.
  \item {\tt STATEFUL} : Entire file defines valid client side states and admissible valid transitions between these states. All functions([ref ServerSpecDFA sendAuth, sendWaitcmd, sendWaitqry, receiveInit, receiveEstablished]) inherited from DFASpec.java defines the valid transition relation between server side states, setting state of server dfa in each of these functions, triggered by \textsf{Message m}. 
  \end{itemize}
\subsection{DFASpec.java}This is an abstract class which defines methods to be implemented by the client and server for the operations to apply at any given state of the protocol ,[TODO]
\begin{itemize}
  \item {\tt STATEFUL}
\end{itemize}

  \section{PDU} All files under this folder represents the message implementation used in protocol. They are received and sent accordingly with the Client/Server's state.

     \subsection{Message.java} Entire file includes definitions of all protocol messages. All messages passed to/from by Client/Server are encapsulated in a Message object.
     \begin{itemize}
     \item {\tt SERVICE}:  Messages are core components of the protocol service in a way that they are used in initiating a connection, initializing states of components of the automobile at the client/server side ex: Command / ACK  etc.
     \item {\tt STATEFUL} : The Message objects are the arrows in the DFA, i.e. the operations by which state transitions in the protocol DFA are applied.
     \end{itemize}
\subsubsection{AckMessage.java}
     \begin{itemize}
     \item {\tt SERVICE} [TODO]
     \item {\tt STATEFUL} [TODO]
     \end{itemize}
       
     \subsubsection{PermanentErrorMessage.java}
     \begin{itemize}
     \item {\tt SERVICE} [TODO]
     \item {\tt STATEFUL} [TODO]
     \end{itemize}

     \subsubsection{QueryResultMessage.java}
     \begin{itemize}
     \item {\tt SERVICE} [TODO]
     \item {\tt STATEFUL} [TODO]
     \end{itemize}

     \subsubsection{RequestReceivedMessage.java}
     \begin{itemize}
     \item {\tt SERVICE} [TODO]
     \item {\tt STATEFUL} [TODO]
     \end{itemize}

     \subsubsection{TemporaryMessage.java}
     \begin{itemize}
     \item {\tt SERVICE} [TODO]
     \item {\tt STATEFUL} [TOOD]
     \end{itemize}

     \subsubsection{TerminationMessage.java}
     \begin{itemize}
     \item {\tt SERVICE} [TODO]
     \item {\tt STATEFUL} [TODO]
     \end{itemize}

     \subsubsection{UserAuthenMessage.java}
     \begin{itemize}
     \item {\tt SERVICE} [TODO]
     \item {\tt STATEFUL} [TODO]
     \end{itemize}

    \subsubsection{UtilityControlReqMessage.java}
     \begin{itemize}
     \item {\tt SERVICE} [TODO]
     \item {\tt STATEFUL} [TODO]
     \end{itemize}

     \subsubsection{UtilityStateQueryMessage.java}
     \begin{itemize}
     \item {\tt SERVICE} [TODO]
     \item {\tt STATEFUL} [TODO]
     \end{itemize} 
     
    \subsection{MessageFactory.java}
     \begin{itemize}
     \item {\tt SERVICE} [TODO]
     \end{itemize}
     
     \subsection{Chunk.java} This file includes abstract definitions for the most basic unit of Message which is core components of protocol's service.
     \begin{itemize}
     \item {\tt SERVICE}: Entire file 
     \end{itemize}

    \subsubsection{ContentChunk.java}This file includes definitions for one type of basic unit of a message of which content of the message comprises.
    \begin{itemize}
     \item {\tt SERVICE}: The relevant requirement the entire file satisfies is partly {\tt SERVICE} because it is one type of basic units of Messages.
     \end{itemize}

    \subsubsection{HeaderChunk.java}This file includes definitions for one type of basic unit of a message of which content of the message comprises.
    \begin{itemize}
    \item {\tt SERVICE} Entire file includes definitions for one type of basic unit of a message which represents header of the file.
      \item {\tt STATEFUL} This file defines the Message Type  which is partly a part of being statefulness. The reason is that Message Type is tightly coupled with Client/Server DFAs' state [ref to one for the switch case in message] when they are processing the message taken from other end. [ref HeaderChunk.java 6]
     \end{itemize}   

  \section{Components} All files under this folder serves for the implementation of representation of an automobile, components of the automobile and commands that are applied on the status of these components. All of them satisfy the {\tt SERVICE} requirement.
  \begin{itemize}
  \item Automobile.java : This file includes the representation of the Automobile object.
  \item Command.java : Commands are operations applied on states of components of an automobile. Command objects are generated from Message objects, [ref to Command.java 27-28] and applied on components of an automobile. It is part of the protocol's service.
  \item Component.java :  This file includes abstract class for a component that can be part of an automobile controlled by the protocol. It is part of the protocol's service because Component.java specifies that each component sub type, ex:AC.java, needs to implement \textsf{applyCommand}, [ref to lines applyCommand], which enables applying a command to change component's state of attributes [ref to 54 - 60 AC.java]. 
  \item ComponentType.java : 
  \item Factory.java : This file includes instantiation of couple of objects (User, Automobile, Components etc.) to demonstrate that protocol is servicing properly.
  \item Query.java : This file includes implementation of read-only operation to check the status of an automobile's components. This is part of the the protocol's service.
  \item AC.java : This is representation of one of the components that an automobile can have. It is sub type of the Component.java
  \end{itemize}
  
  \section{Users}
  \subsection{User.java} This file includes representation of user which is one of the core parts of the protocol service in a way that 
  \begin{itemize}
  \item {\tt SERVICE}: Authentication of an user with user name and password, [ref 4-5 User.java],  is one important part of the protocol's service. Each Client object is created with its user name and password, [ref 41-70 Client.java] and creates/sends authenticate message, UserAuthMessage.java [ref 112 Client.java], 
    \end{itemize}
  
\end{document}

\begin{comment}
  We incorporate the analysis of these requirements to each file in code base as header explanations. The structure of the header is :

    \begin{adjustbox}{width=\textwidth,height=\someheight,keepaspectratio}
    \begin{lstlisting}
  /* =============================================================================
 * CS544 - Computer Networks - Drexel University, Spring 2017
 * Protocol Implementation: Remote Automobile Utility Control
 * Group 2:
 * - Ismail Kuru
 * - Max Mattes
 * - Lewis Cannalongo
 *
 * File name: Message.java
 * 
 * Definition: Definitions of all protocol messages. All messages passed
 * from one end to another are encapsulated in a Message object.
 * 
 * Requirements:
 * - STATEFUL : the Message objects are the arrows in the DFA, i.e. the operations
 *   by which state transitions in the protocol DFA are applied.
 * - SERVICE : the messages are an important part in the protocol service
 *   definition: initiating a connection, initializing states of components of the
 *   automobile at the client/server side ex: Command / ACK  etc.
 * 
 * ==============================================================================
 */      
    \end{lstlisting}
    \end{adjustbox}


In addition, If specific code segments address relates the implementation to a specific requirement, a comment-out section including the explanation of this relation is added.

  \end{comment}
