
\section{Message Definition -- PDU}
\label{sec:pdus}

In this section, we discuss the details of the \textsf{RAUC} protocol.

 \subsection{Addressing}
\label{sec:pdus:addr}
\textsf{RAUC} protocol uses TCP/IP for transportation. It uses stream-oriented channels to send messages with no fixed sizes. [Note: We can customize port numbers to for specific task done by a device on the car. For example: We may want to assign base\_port\_number + device\_port\_index ex: 1010+1 = 1011 port no for air conditioner. ].

Based on port configuration we explained, a TCP connection is established to the \textsf{CCS} IP address and port number provided.

\subsection{Flow Control}
\label{sec:pdus:flow}
Flow control is dispatched to TCP/IP layer. [Note: As future extension, we may mention that we can enhance the flow control with adding asychrony in communication such as when client sends a message  and returns but sender send feedback later on. This reduces the congestion of network. In addition, we may buffer requests and flush them to server to save reduce communication overhead for each command.]

\subsubsection{Handshake}
\label{sec:pdus:pdu:hs}
All the handshaking is handled with [TODO]

\subsection{PDU (Message) Definitions}
RAUC uses a single generic, variable sized PDU comprised of two parts: A single HEADER CHUNK and an ordered set of up to 255 CONTENT CHUNKs. The PDU is delineated by the end of the final CONTENT CHUNK in the message. Determining the position of the end of the last CONTENT CHUNK is defined below. This same PDU is used for both client-to-server and server-to-client messages.

\[\textsf{\textbf{MESSAGE:} [HEADER CHUNK} : \textsf{[CONTENT CHUNK] [...]]}\]

The HEADER CHUNK is a special, fixed-size chunk that contains two one-byte elements. The first byte specifies the MESSAGE TYPE being transmitted and the second byte contains the number of content chunks (CHUNK COUNT) contained in the message. This means that RAUC supports up to 256 unique message types and The number of content chunks in a given message can range from 0 to 255.

\[\textsf{\textbf{HEADER CHUNK:} [MESSAGE TYPE} : \textsf{CHUNK COUNT]}\]

CONTENT CHUNKs are generically defined and dynamically sized message elements. A content chunk has two parts: SIZE and DATA. SIZE is a single byte that defines how many bytes make up the data portion of the chunk. DATA is the actual data content of the chunk that is meaningful to the application. RAUC clients \textbf{MUST} size the data sent in each content chunk it sends out; relying on default implementations (such as setting all usernames to 32 bytes long and setting all unused high-order bytes to 0) is not advised.

\[\textsf{\textbf{CONTENT CHUNK:} [SIZE} : \textsf{DATA]}\]



\subsubsection{Minimum Required Implementation}
All RAUC implementations \textbf{MUST} implement the following  messages, as well as the two error messages, to be considered fully-functioning. In table \ref{tab:coml}, we list the message types and their usage. We basically assign a type code to each message type so that we can identify what sort of message is sent/recevived. We structure our messages in a way that we can support variable length messages. We keep track of number of required content chunks of for each message. 

\label{sec:pdus:pdu:c_to_s}
\
\begin{table}[ht!]
  \centering
\resizebox{\textwidth}{!}{
\begin{tabular}{l l l l}
\hline
\textbf{Message Type} & \textbf{Message Type Code} &\textbf{Required Content Chunks} &   \textbf{Usage}  \\
\hline

\hline
\textsf{User Authenticatoin}  		& \textsf{3}		    & \textsf{2} 	& \textsf{Client Handshake Message}   \\
\hline
       \textsf{Authenticatoin Acknowledgement}    & \textsf{4}       & \textsf{[0 | 1]}  & \textsf{Server Handshake Message}  \\
\hline
     \textsf{Utility Control Request}             & \textsf{5}  & \textsf{4} & \textsf{Client Command to Server}\\
\hline
  \textsf{Request Received}             & \textsf{15}   & \textsf{[0 | 1]} & \textsf{Server Response to UCR}\\
\hline
  \textsf{Utility State Query}             & \textsf{6}   & \textsf{[0 | 1 | 2 | 3 | 4 ]} & \textsf{Client Command to Server}\\
\hline
  \textsf{Query Result}             & \textsf{16}  & \textsf{1} & \textsf{Server Response to Query}\\
\hline
  \textsf{Permanent Error}             & \textsf{255}   & \textsf{0} & \textsf{Error Message}\\
\hline
  \textsf{Temporary Error}             & \textsf{254}   & \textsf{0} & \textsf{Error Message}\\
\hline
\end{tabular}}

\caption{Message Types}
\label{tab:coml}
\end{table}

\paragraph{User Authentication}: After establishing a connection, ACS servers \textbf{MUST NOT} accept any commands other than User Authentication(header [3:2]) or CLOSE (header [9:0])  before first authenticating the client. Once authenticated, clients can issue further commands. The User Authentication command is a 3-chunk message made up of the header and two content chunks: USERNAME and PASSWORD.  With "..." representing user data, the following is an example of the User Authentication command with USERNAME that is 32 bytes long and a PASSWORD that is 256 bytes long:

\begin{center}
{\textsf{[3:2] : [USERNAME] [PASSWORD]}}

{\textsf{[3:2] : [32: ... ] [256: ... ]}}
\end{center}

\paragraph{Authentication Acknowledgement}: Authentication Acknowledgement (AA) (header [4:0-*]) commands are sent by ACS servers ONLY after successful user authentication. This message lets the client know that the server is ready to begin receiving Utility Control Request or Utility State Query messages. ACS servers \textbf{MAY} use Authentication Acknowledgement commands to transmit additional information such as protocol version, supported extensions with an extended content chunk count. The message header being set to 4 is enough for clients to know that authentication was successful and that the server is listening for commands. Two examples appear below.

\begin{center}
{\textsf{[4:0] : []}}

%\\[1\baselineskip]
{\textsf{[4:1] : [SERVER VERSION]}}

{\textsf{[4:1] : [1 : ...]}}
\end{center}

\paragraph{Utility Control Request}: A Utility Control Request (UCR) (header [5:4]) represents a command that a smart device wishes to ultimately forward to an automobile connected to the ACS server. The ACS server \textbf{MUST} acknowledge receipt of a Utility Control Request message but \textbf{MAY} choose whether or not to acknowledge if the command was successfully forwarded. A UCR is a 5-chunk message made up of the header and four content chunks: AUTOID (unique identifier of the automobile), COMP (the specific component on the automobile to access), ATTR (the attribute of the component we wish to modify), and VAL (the value we wish to set the attribute to). Using some sample user data, the following is an example of the User Authentication command:

\begin{center}
{\textsf{[5:4] : [AUTOID] [COMP] [ATTR] [VAL]}}

{\textsf{[5:4] : [red car] [radio] [volume] [5]}}
\end{center}

\paragraph{Request Received}: A Command Received (CR) (header [15:0-1]) message is sent to the client when the server successfully receives a Utility Control Request command. The message does not contain any information
about whether or not such commands were forwarded; only that the command was valid and accepted by the server. Command received messages may contain 0 or 1 content chunks. ACS servers \textbf{MAY} send 1-byte content chunks to verify whether or not a command was successfully forwarded to an automobile. A content chunk with a DATA portion of all 1s indicates success while a DATA portion of all 0s indicates failure.

\begin{center}
{\textsf{[15:0] : []}}

%\\[1\baselineskip]
{\textsf{[15:1] : [SUCCESS]}}

{\textsf{[15:1] : [1 : 1111 1111]}}
\end{center}

\paragraph{Utility State Query}: A Utility State Query (USQ) (header [6:0-4]) message alerts the server that the device would like some information returned. This command has 0 to 4 CONTENT CHUNKs depending on the nature of the query, however a hierarchical rule must be followed when providing arguments. The arguments AUTO, COMP, ATTR, and VAL are detailed below.

If no arguments are provided, the server returns a list that uniquely identifies each automobile which has been registered to this account that is currently connected to the ACS server.  This list essentially provides a list of automobiles that the client can interact with at the present time.

The AUTO argument (content chunk 1) is the unique identifier for a specific automobile which this account has registered to itself (i.e. from the previously mentioned list). If this argument is provided, the server returns a list of all component identifiers which the client may interact with on the specific automobile.

The COMP argument may only be provided if the AUTO argument has been provided. COMP is the identifier which identifies a specific component on AUTO. If this argument is provided, the server returns a list of all attributes available for modification on the indicated component on the indicated auto.

The ATTR argument may only be provided if the AUTO and COMP arguments have been provided. ATTR is the attribute of the specific component on a specific automobile. If this argument is provided, the server returns a list of all values which are valid for the given attribute on the specific component.

The VAL argument may only be provided if the AUTO, COMP, and ATTR arguments have been provided. It is a dummy content chunk that instructs the server to return  is the currently-set value of the ATTR on the specific COMP for the specific AUTO provided.

The following message would provide a list of all radio presets [ATTR] saved on [RED CAR]'s [RADIO] component:

\begin{center}
{\textsf{[6:3] : [AUTOID] [COMP] [ATTR]}}

{\textsf{[6:3] : [RED CAR] [RADIO] [PRESETS]}}
\end{center}

\paragraph{Query Result}: A Query Result (QR) (header [16:1]) message is sent by the server in response to a Utility State Query. Query Result messages contain a  header chunk and only one content chunk: the result list of a successful query. The content chunk byte-length will naturally vary depending on the result set returned to the client. The following represents a Query Result message for a one-argument Utility State Query with a 64 byte content chunk.

\begin{center}
{\textsf{[16:1] : [REGISTERED VEHICLES]}}

{\textsf{16:1] : [64 : ... }}
\end{center}


\paragraph{Termination}: Termination (header [9:0]) messages may be sent by both client and server. Once command for termination is sent by any of two, all resources being used for the connection is properly finalized and connection is terminated.

\begin{center}
{\textsf{[9:0] : []}}
\end{center}



\subsection{Error Control}
\label{sec:pdus:err}
RAUC supports two types of error messages: Permanent and Temporary.

\paragraph{Permanent Error}: (header [255:0]) Permanent errors are sent by the server to the client and instructs the client that the same message sent in the current state will never be accepted. Permanent errors may be sent under these conditions:
\begin{itemize}
\item Malformed message received from client.
\item Information provided with User Authentication command not verified.
\item Message received for an invalid command (eg. User Authentication message sent at any point after authentication handshake)
  \end{itemize}
  
\begin{center}
{\textsf{[255:0] : []}}
\end{center}

\paragraph{Temporary Error}: (header [254:0]) Temporary error messages are sent by the server when a command is not received by the server because of a temporal machine state problem. Temporary errors indicate that the client sent a valid command which can be tried again at a later time without modification. The most common reason that a server will send a temporary error message is due to a lack of resources to fulfill a command sent by the client.

\begin{center}
{\textsf{[254:0] : []}}
\end{center}
