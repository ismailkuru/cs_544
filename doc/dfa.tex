\section{DFA}
\label{sec:dfa}

\begin{tikzpicture}[->,>=stealth',shorten >=1pt,auto,node distance=4.8cm,
                    semithick]
  \tikzstyle{every state}=[draw=black,shape=rectangle]

  \node[initial,state] (A)                    {CLOSED};
  \node[state]         (B) [right of=A]       {INIT};
  \node[state]         (C) [right of=B] {AUTH};
  \node[state]         (D) [below left of=C]       {ESTABLISHED};
  \node[state]         (F) [left of=D]        {WAITCMD};
  \node[state]         (G) [right of=D]       {WAITQRY};
  \path (A) edge node {TLS connection} (B)
        (B) edge [bend left] node[above]       {auth} (C)
        (C) edge node[sloped, align=left, swap] {success} (D)
            edge [bend left] node[above] {failure} (B)
        (D) edge [right] node[above] {command}   (F)
            edge [left] node[above]  {query} (G)
            edge node[sloped, swap]  {shutdown} (A)
        (F) edge [bend right] node[below, align=center] {success\\OR failure\\OR error}   (D)
        (G) edge [bend left]  node[below, align=center] {info\\OR error}   (D);
\end{tikzpicture}

In addition to the information pictured in the DFA, the shutdown command can be sent from any state. Sending or receiving a shutdown results in termination of the TLS session and moving to the CLOSED state. The implementation can choose whether to send an error in any state where an invalid or malformed message is received, and the recipient can decide whether to send a new message or send a shutdown to terminate the connection. The server should also maintain a \texttt{retries} count in the INIT state to count the number of failures sent, and after a configureable \texttt{max\textunderscore retries}, it should sent a shutdown.