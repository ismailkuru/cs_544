\section{Service Description}
\label{sec:serv_desc}

The Remote Automobile Utility Control \textbf{(RAUC)} protocol specifies communication between user-controlled devices and the Automobile Control Server \textbf{(ACS)} through variable-sized messages sent across a TCP connection with all communications encrypted through TLS. These messages include commands for the ACS to forward to the automobile, and queries against automobile component state. RAUC connections require clients to authenticate themselves with the ACS before any commands or queries may be issued. This protocol assumes the user has already registered their car with the ACS, and that they have a valid user account authorized to access the vehicle in question. It is also assumes that the vehicle is either always connected to the internet, or that commands and queries will be queued until the vehicle is reachable, but the exact details are handled by the ACS and beyond the scope of the protocol itself.

The protocol itself provides a secure and extensible way to issue commands to a fleet of vehicles. The PDUs are flexible enough to allow for query and control of a variety of components and remain extensible into the potential future of car development beyond the different types of components which exist today.
